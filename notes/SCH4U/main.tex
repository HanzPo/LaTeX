\documentclass[letterpaper]{article}

\usepackage[english]{babel}
\usepackage[utf8x]{inputenc}
\usepackage[T1]{fontenc}
\usepackage{amsmath}
\usepackage{graphicx}
\usepackage{lewis}
\usepackage{chemfig}

\title{\huge Structures and Properties of Matter}
\author{Hanz Po}
\date{\today}

\begin{document}


\maketitle



\section{Types of Chemical Bonds}

\subsection{Lewis Theory of Bonding}

The Lewis Theory of Bonding can be described with the following characteristics:
\begin{itemize}
  \item Atoms and ions are stable if they have a full valence shell of electrons (noble gas configuration)
  \item Electrons are most stable when they are paired
  \item Atoms form chemical bonds to achieve a full valence shell of electrons. This may be achieved in two ways:
  \begin{enumerate}
    \item An exchange of electrons between metal and non-metal Atoms
    \item Sharing of electrons between non-metal atoms
  \end{enumerate}
\end{itemize}

\subsection{Lewis Diagrams}

Lewis diagrams are simplified forms of Bohr-Rutherford diagrams. The chemical symbol represents the nucleus and core electrons, while dots around the symbol represent the valence electrons.

\lewis{O}{.}{.}{.}{.}{.}{.}{.}{}

\end{document}