\documentclass[letterpaper]{article}

\usepackage[english]{babel}
\usepackage[utf8x]{inputenc}
\usepackage[T1]{fontenc}
\usepackage{amsmath}
\usepackage{graphicx}
\usepackage{pgfplots}

\title{\huge Trigonometric Identities and Equations Notes}
\author{Hanz Po}
\date{\today}

\begin{document}


\maketitle

\section{Equivalent Trigonometric Functions}

There are a number of ratios which evalute to equal values when provided with different angles.

\begin{tikzpicture}[>=stealth]
  \begin{axis}[
      xmin=-4,xmax=4,
      ymin=-2,ymax=2,
      axis x line=middle,
      axis y line=middle,
      axis line style=<->,
      xlabel={$x$},
      ylabel={$y$},
      ]
      \addplot[no marks,blue,<->] expression[domain=-pi:pi,samples=100]{sin(deg(x))} 
                  node[pos=0.65,anchor=south west]{$y=\sin(x)}$}; 
  \end{axis}
\end{tikzpicture}

\end{document}